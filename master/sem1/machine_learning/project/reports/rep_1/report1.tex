\documentclass[a4paper, 11pt]{article}
\usepackage{geometry}
\usepackage{url}
\usepackage{color}


\linespread{1}

\geometry{a4paper,top=3cm,left=3cm,right=2.5cm,bottom=2cm}

\usepackage{hyperref}
\hypersetup{colorlinks,linkcolor=black,citecolor=blue}

\title{\textbf{Laboratory assignment} \\[1ex] \large \textbf{Component} {1}}

\author{\textbf{Authors:} {Ichim Stefan, Mirt Leonard}\\ \textbf{Group:} {246/1}}

\begin{document}

\maketitle

\section{Task 1 - Unsupervised Learning}

\subsection{Problem Definition}

\begin{itemize}
  \item{
    We want to create an algorithm that could assist us in 
    observing similarities between different housings in California,
    observations which could have been missed by market analysts.
  }
\end{itemize}

\subsection{Problem Specification}

\subsubsection*{Inputs} 
  Tabluar data from the "California Housing Prices" dataset.
  Features which will be used for the pattern learning
  consist of: longitude, latitude, housing median age, total rooms,
  total bedrooms, population, households, median income and ocean 
  proximity. The median house value will act as the target value 
  and will be used to evaluate if patterns from the same clusters
  have similar values.

\subsubsection*{Preconditions} 
  Pattern features will need to be scaled for the algorithm
  to work with numbers from a smaller interval.
  $$
  x = \frac{x - mean(x)}{std(x)}
  $$

\subsubsection*{Outputs} 
  The unsupervised algorithm will indicate which cluster a pattern
  will be part of.
  
\subsubsection*{Postconditions} 
  A natural number, representing the number of the cluster.

\subsection{Learning Task Specification}

\subsubsection*{Task}
Create appropriate clusters for various patterns in the dataset.

\subsubsection*{Performance}
The algorithm will be evaluated using both external and internal
performance measures. According to these results, hyperparameters
will suffer changes so that the algorithm reaches higher results.

For internal measures, silhouette score and Calinski-Harabasz Index
can be used. The first one measures how similar an object is to its own cluster
compared to other clusters, and the latter measures the ratio
of between-cluster dispersion to within-cluster dispersion.

An external measure could be represented by a "feature hold-out" technique,
where we choose to exclude a feature from the pattern when building our clusters,
and then later on using that feature to examine how well the created clusters
allign with it.

\subsubsection*{Experience}
The experience consists of patterns represented by the entries
of the dataset.

\end{document}
