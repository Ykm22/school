\documentclass[0.5p,a4paper]{article}

\usepackage[bookmarks=false]{hyperref}
    \hypersetup{colorlinks,
      linkcolor=blue,
      citecolor=blue,
      urlcolor=blue}

\begin{document}

\author{Ichim Stefan}
\title{\textbf{ML - Week 4 Assignment}}
\maketitle

\textbf{- Topic Description} \\
Bayesian learning represents a branch of machine learning whose
inference has foundations in statistics. The algorithm 
initially receives already known information regarding 
the state of the world. It makes probabilistic assumptions
regarding our hypothesis of interest and as it receives new
evidence, these assumptions suffer modifications. The whole
process happens using the Bayes theorem of probabability.

%- difference between other networks
Since their existence, they have proven to be competitive
with both traditional (support vector machines, decision
trees) and novel (deep neural networks) learning techniques.
These appear in various fields such as: medical diagnosis, 
finance and natural language processing. It is widely known
most of these approaches use a 'black-box' paradigm,
where models improve unbeknownst to the designers,
however, bayesian networks' inference can in fact be understood
given it is based on a statistical formula.

%- how it is used in medical diagnosis
The chosen topic for this research report will surround how
bayesian networks approach medical diagnosis and their
advantages and drawbacks compared with the previously
mentioned techniques. More specifically, these networks
are used in diagnosis ranging from cardiovascular diseases,
pregnancy disorders to even psychological and psychiatric
disorders. The applications are far and wide, which only
serve to underline the importance and versatility of these
kinds of networks.

\textbf{- Research Report Title} \\
Bayesian Learning in Medical Diagnosis test

\textbf{- References} \\
  - Ananda Rao, A., Awale, M., Davis, S., 2023. Medical diagnosis reimagined as a process of bayesian reasoning and elimination. Cureus 15, e45097. \\
  - Elsayad, A.M., Fakhr, M., 2015. Diagnosis of cardiovascular diseases with bayesian classifiers. J. Comput. Sci. 11, 274–282. \\
  - Julia Flores, M., Nicholson, A.E., Brunskill, A., Korb, K.B., Mascaro, S., 2011. Incorporating expert knowledge when learning bayesian network structure: A medical case study. Artificial Intelligence in Medicine 53, 181-204. \\
  - McLachlan, S., Dube, K., Hitman, G.A., Fenton, N.E., Kyrimi, E., 2020. Bayesian networks in healthcare: Distribution by medical condition. Artificial Intelligence in Medicine 107, 101912. \\
  - Nour, M., Cömert, Z., Polat, K., 2020. A novel medical diagnosis model for covid-19 infection detection based on deep features and bayesian optimization. Applied Soft Computing 97, 106580. \\

\textbf{- Presentation Date} - Week 9 lecture hours, Tuesday 16-18, C310

\end{document}
