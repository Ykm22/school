\documentclass{article}
\usepackage[margin=1in]{geometry} % For better margins
\usepackage{graphicx} % Required for \rotatebox command
\usepackage{booktabs} % For better table formatting
\usepackage{caption} % For table caption customization
\usepackage{enumitem}

\title{Exercise sheet Nr. 1}
\author{Ichim Stefan - ICA - 246/1}
\date{\today}

\begin{document}

\maketitle

\left(\textbf{G 1}\right)

\\

Regard the following formal context K, given as a cross table:

\begin{table}[htbp]
\centering
\label{tab:species}
\begin{tabular}{|l||c|c|c|c|c|c|c|c|c|}
\hline
  & \rotatebox{90}{needs water to live} & \rotatebox{90}{lives in water} & \rotatebox{90}{lives on land} & \rotatebox{90}{needs chlorophyll to produce food} & \rotatebox{90}{two seed leaves} & \rotatebox{90}{one seed leaf} & \rotatebox{90}{can move around} & \rotatebox{90}{has limbs} & \rotatebox{90}{suckles its offspring} \\
\hline\hline
  Leech & x & x &  &  &  &  & x &  & \\
\hline
  Bream & x & x &  &  &  &  & x & x & \\
\hline
Frog & x & x & x &  &  &  & x & x & x \\
\hline
Spike-Weed & x & x &  & x &  & x &  &  & \\
\hline
  Reed & x & x & x & x & & x &  &  & \\
\hline
  Bean & x &  & x & x & x &  &  &  & \\
\hline
  Maize & x &  & x & x &  & x &  &  & \\
\hline
\end{tabular}
\end{table}

\begin{enumerate}[label=\alph*)]
  \item Specify the following sets:
    \begin{enumerate}[label=\arabic*.]
      \item \{Bean\}' = \{needs water to live, lives on land, needs chlorophyll to produce food, two seed leaves\}
      \item \{lives on land\}' = \{Frog, Reed, Bean, Maize\}
      \item \{two seed leaves\}'' = \{Bean\}' = 1)
      \item \{Frog, Maize\}' = \{needs water to live, lives on land\}
      \item \{needs clorophyll to produce food, can move around\}' = \{\}
      \item \{lives in water, lives on land\}' = \{Frog, Reed\}
      \item \{needs clorophyll to produce food, can move around\}' = 5)
    \end{enumerate}
  \item Extend $K$ with both an object and an attribute
    \begin{itemize}
      \item $G \cup \{Lizard\},\ M \cup \{has\ scales\}$
        \begin{table}[htbp]
        \centering
        \label{tab:species_2}
        \begin{tabular}{|l||c|c|c|c|c|c|c|c|c|c|}
        \hline
          & \rotatebox{90}{needs water to live} & \rotatebox{90}{lives in water} & \rotatebox{90}{lives on land} & \rotatebox{90}{needs chlorophyll to produce food} & \rotatebox{90}{two seed leaves} & \rotatebox{90}{one seed leaf} & \rotatebox{90}{can move around} & \rotatebox{90}{has limbs} & \rotatebox{90}{suckles its offspring} & \rotatebox{90}{has scales} \\
        \hline\hline
          Leech & x & x &  &  &  &  & x &  &  & \\
        \hline
          Bream & x & x &  &  &  &  & x & x &  & x\\
        \hline
        Frog & x & x & x &  &  &  & x & x & x & \\
        \hline
        Spike-Weed & x & x &  & x &  & x &  &  &  & \\
        \hline
          Reed & x & x & x & x & & x &  &  &  & \\
        \hline
          Bean & x &  & x & x & x &  &  &  &  & \\
        \hline
          Maize & x &  & x & x &  & x &  &  &  & \\
        \hline
          Lizard & x &  & x &  &  &  & x & x &  & x\\
        \hline
      \end{tabular}
      \end{table}
    \end{itemize}
\end{emnumerate}

\newpage

\left(\textbf{G 2}\right)

Consider the formal context from Lecture 1. Use \textbf{ConExp} and \textbf{FCA Tools Bundle} to determine the set of concepts and to draw the concept lattices.

Using the FCA Tools Bundle from https://fca-tools-bundle.com/, the following concepts were generated:

\begin{table}[htbp]
\centering
\begin{tabular}{|l|l|}
\hline
\textbf{Concept} & \textbf{Attribute} \\
\hline\hline
Leech & needs water to live \\
\hline
Leech & lives in water \\
\hline
Leech & can move around \\
\hline
Bream & needs water to live \\
\hline
Bream & lives in water \\
\hline
Bream & can move around \\
\hline
Bream & has limbs \\
\hline
Frog & needs water to live \\
\hline
Frog & lives in water \\
\hline
Frog & lives on land \\
\hline
Frog & can move around \\
\hline
Frog & has limbs \\
\hline
Frog & suckles its offspring \\
\hline
Spike-Weed & needs water to live \\
\hline
Spike-Weed & lives in water \\
\hline
Spike-Weed & needs chlorophyll to produce food \\
\hline
Spike-Weed & one seed leaf \\
\hline
Reed & needs water to live \\
\hline
Reed & lives in water \\
\hline
Reed & lives on land \\
\hline
Reed & needs chlorophyll to produce food \\
\hline
Reed & one seed leaf \\
\hline
Bean & needs water to live \\
\hline
Bean & lives on land \\
\hline
Bean & needs chlorophyll to produce food \\
\hline
Bean & two seed leaves \\
\hline
Maize & needs water to live \\
\hline
Maize & lives on land \\
\hline
Maize & can move around \\
\hline
Maize & has limbs \\
\hline
\end{tabular}
\label{tab:organism_characteristics}
\end{table}

\begin{figure}[htbp]
\centering
\includegraphics[width=0.8\textwidth]{FCA_Tools_Bundle_ConceptLattice.png}
\caption{Concept lattice generated by FCA Tools Bundle}
\label{fig}
\end{figure}

\newpage

\left(\textbf{G 3}\right)

\begin{enumerate}[label=\alph*)]
  \item Recall: how is the derivation operator $(\cdot)'$ defined?
    \begin{itemize}
      \item The derivation operator can be applied on a set of objects or a set of attributes.
      \item When it is applied on a set of \textbf{objects}, the result is represented by the set of \textbf{attributes} all objects in the first set have in common.
      \item When it is applied on a set of \textbf{attributes}, the result is represented by the set of \textbf{objects} which have all the attributes from the original set.
    \end{itemize}

  \item Let $K = \left(G, M, I\right)$ be a formal context and let $A, B \subseteq G$. Prove the following statements:
   \begin{enumerate}[label=\arabic*.]
     \item $A \subseteq B$ implies $B' \subseteq A'$
       \begin{itemize}
         \item if $A \subseteq B$, then for any attribute $m \in B'$, we know that $\forall g \in B: (g,m) \in I$
         \item since $A \subseteq B$, every element in $A$ is also in $B$
         \item therefore, for any $g \in A$, we have $g \in B$, which means $(g,m) \in I$
         \item this shows that $\forall g \in A: (g,m) \in I$, which is exactly the definition of $m \in A'$
         \item hence, if $m \in B'$, then $m \in A'$, which proves that $B' \subseteq A'$
       \end{itemize}
     \item $A \subseteq A''$
       \begin{itemize}
         \item for any $g \in A$, we need to show that $g \in A''$
         \item by definition, $A'' = (A')'$, which means $A'' = \{g \in G \mid \forall m \in A': (g,m) \in I\}$
         \item for any $g \in A$ and any $m \in A'$, we know that $(g,m) \in I$ by the definition of $A'$
         \item therefore, every $g \in A$ satisfies the condition for being in $A''$
         \item hence, $A \subseteq A''$
       \end{itemize}
     \item $A' = A'''$
       \begin{itemize}
         \item we'll prove both inclusions: $A' \subseteq A'''$ and $A''' \subseteq A'$
         \item from (2), we know that $A' \subseteq A'''$ because we can substitute $A'$ for $A$ in the statement $A \subseteq A''$
         \item for the other direction, let $m \in A'''$, then $\forall g \in A'': (g,m) \in I$
         \item from (2), we know $A \subseteq A''$, so for every $g \in A$, we have $g \in A''$
         \item therefore, $(g,m) \in I$ for all $g \in A$
         \item this means $m \in A'$ by definition
         \item thus, $A''' \subseteq A'$
         \item combining both inclusions, we have $A' = A'''$
       \end{itemize}
     \item For $C \subseteq G$ and $D \subseteq M$ holds: $(C, D)$ is a formal concept if and only if there is some $E \subseteq G$ such that $C = E''$ and $D = E'$
       \begin{itemize}
         \item first, we'll prove the forward direction: if $(C, D)$ is a formal concept, then there exists $E \subseteq G$ such that $C = E''$ and $D = E'$
         \item by definition, a formal concept $(C, D)$ satisfies $C' = D$ and $D' = C$
         \item let $E = C$, then $E' = C' = D$ and $E'' = D' = C$
         \item therefore, there exists $E$ (namely $C$ itself) such that $C = E''$ and $D = E'$
         \item for the reverse direction, assume there exists $E \subseteq G$ such that $C = E''$ and $D = E'$
         \item we need to show that $(C, D)$ is a formal concept, which means $C' = D$ and $D' = C$
         \item since $D = E'$, we have $D' = (E')' = E''$ by definition
         \item since $C = E''$, we have $D' = E'' = C$
         \item now we need to show $C' = D$
         \item since $C = E''$, by the property in (3), we have $C' = (E'')' = E'''$
         \item again from (3), we know $E' = E'''$
         \item therefore, $C' = E''' = E' = D$
         \item thus, $(C, D)$ is a formal concept if and only if there exists $E \subseteq G$ such that $C = E''$ and $D = E'$
       \end{itemize}
   \end{enumerate}
\end{enumerate}

\end{document}
