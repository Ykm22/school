\documentclass[a4paper]{article}

\title{
    Alzheimer's Disease Prediction \\
    ELL - Lab 2
}
\author{Ichim Ștefan}
\date{1st April 2024}

\begin{document}

\maketitle

\tableofcontents

\newpage

\section{Theoretic}

\subsection{Abstract}
Summarize the future sections in this research article.

\subsection{Introduction}
Elaborate the background of the related domains which the article tackles - Deep Learning and Alzheimer's Disease.

\subsection{Related Work}
Enumerate approaches from different state-of-the-art articles in line with the topic, their advantages, disadvantages and
further improvements.

\subsection{Dataset}
Describe possible datasets to be used, their origin, availability and differences.

\subsection{Proposed Approach}
Detail the researched techniques, starting from the reasoning, then presenting them at a lower level.

Compare them to other related works.

Furthermore, explain the methods of evaluation and loss functions tested, which ones were chosen and for what reason.

\subsection{Results and Experiments}
List the results of various techniques and approaches using graphs and tables.

\subsection{Discussion}
Describe the advantages and disadvantages of the paper and its applications.

\subsection{Conclusions and future work}
A final summarization of the paper, which intends to remind the reader of the previous sections,
as well as put forward future improvements revealed throughout researching the topic.

\subsection{References}
An ordered list of articles which will have been referenced throughout this research paper.

\section{Code}

\subsection{Libraries}
Detail which libraries were used and why, drawbacks and advantages over others.

\subsection{Dataset handling}
Explain, using code, how the dataset/s were handled, pre-processed, the specifities of each of them.

\subsection{Algorithm}
Describe how the algorithm functions piece by piece, starting from network, to loss functions.

\subsection{Results}
Display how the results were calculated, through which methods and their percentages.

\subsection{Own contributions}
Enlighten pieces of code which were personal contributions to the topic and bring forward improvements.

\end{document} % Marks ending of body