\documentclass[3p,times,procedia]{elsarticle}

\flushbottom

%% The `ecrc' package must be called to make the CRC functionality available
\usepackage{ecrc}
\usepackage[bookmarks=false]{hyperref}
    \hypersetup{colorlinks,
      linkcolor=blue,
      citecolor=blue,
      urlcolor=blue}
%\usepackage{amsmath}

\firstpage{1}
\runauth{Author Name}

\usepackage{amssymb}
%% The amsthm package provides extended theorem environments
\usepackage{amsthm}
\usepackage{amsmath}
\usepackage{multirow}
\usepackage{comment}
\usepackage[figuresright]{rotating}

\begin{document}
\begin{frontmatter}

    \dochead{\huge{Tehnici de realizare a sistemelor inteligente\\ (TRSI 2024)}}%
    %% Use \dochead if there is an article header, e.g. \dochead{Short communication}
    %% \dochead can also be used to include a conference title, if directed by the editors
    %% e.g. \dochead{17th International Conference on Dynamical Processes in Excited States of Solids}


    \title{\textbf{Here is the title of the report}}

    \author{Author name}

    \address{Department of Computer Science, Babe\c s-Bolyai University\\1, M. Kogalniceanu Street, 400084, Cluj-Napoca, Romania\\E-mail: .....}

    \begin{abstract}
        %% Text of abstract

    \end{abstract}

    \begin{keyword}
    \end{keyword}
\end{frontmatter}
\section{Introduction}\label{introduction}

%\section{\textcolor{red}{Other sections to be added}}

\cite{2018_Dafonte} \cite{2018_Tolstikhin}

\section{Discussion}\label{discussion}

\section{Conclusions and future work}\label{conclusions}



\footnotesize{

    \bibliographystyle{elsarticle-harv}
    \bibliography{references}

}

\end{document}